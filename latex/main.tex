\documentclass[a4paper]{report}

%====================== PACKAGES ======================

\usepackage[french]{babel}
\usepackage[utf8x]{inputenc}
%pour gérer les positionnement d'images
\usepackage{float}
\usepackage{amsmath}
\usepackage{graphicx}
\usepackage[colorinlistoftodos]{todonotes}
\usepackage{url}
%pour les informations sur un document compilé en PDF et les liens externes / internes
\usepackage{hyperref}
%pour la mise en page des tableaux
\usepackage{array}
\usepackage{tabularx}
%pour utiliser \floatbarrier
%\usepackage{placeins}
%\usepackage{floatrow}
%espacement entre les lignes
\usepackage{setspace}
%modifier la mise en page de l'abstract
\usepackage{abstract}
%police et mise en page (marges) du document
\usepackage[T1]{fontenc}
\usepackage[top=2cm, bottom=2cm, left=2cm, right=2cm]{geometry}
%Pour les galerie d'images
\usepackage{subfig}

%====================== INFORMATION ET REGLES ======================

%rajouter les numérotation pour les \paragraphe et \subparagraphe
\setcounter{secnumdepth}{4}
\setcounter{tocdepth}{4}

\hypersetup{							% Information sur le document
pdfauthor = {Premier Auteur,
			Deuxième Auteur,
			Troisième Auteur,
    		Quatrième Auteur},			% Auteurs
pdftitle = {Nom du Projet -
			Sujet du Projet},			% Titre du document
pdfsubject = {Mémoire de Projet},		% Sujet
pdfkeywords = {Tag1, Tag2, Tag3, ...},	% Mots-clefs
pdfstartview={FitH}}					% ajuste la page à la largueur de l'écran
%pdfcreator = {MikTeX},% Logiciel qui a crée le document
%pdfproducer = {}} % Société avec produit le logiciel

%======================== DEBUT DU DOCUMENT ========================

\begin{document}

%régler l'espacement entre les lignes
\newcommand{\HRule}{\rule{\linewidth}{0.5mm}}

%%%
 % File:     web.tex
 % Author:   Hackademics Forum <hackademicsforum6@gmail.com>
 % Project:  MindMap des vulnérabilités
 % Released: 03/08/2016
%%%

%!TeX root = main.tex
%!TeX encoding = UTF-8
%!TeX program = pdflatex
%!TeX spellcheck = fr_FR

%%%
 % Vulnérabilités WEB
%%%
\chapter{Vulnérabilités WEB}\label{vulnerabilites:web}

Intro vulnérabilités WEB

\section{Structured Query Language Injection (SQLi)}\label{vulnerabilites:web:sqli}

Intro vulnérabilités SQLi

\subsection{SQLi Simple}\label{vulnerabilites:web:sqli:simple}

...

\subsection{SQLi à l'aveugle (Blind)}\label{vulnerabilites:web:sqli:blind}

...

\section{Cross-Site Scripting (XSS)}\label{vulnerabilites:web:xss}

Intro vulnérabilités XSS

\subsection{XSS local (DOM Based)}\label{vulnerabilites:web:xss:dom}

...

\subsection{XSS stocké (Stored)}\label{vulnerabilites:web:xss:stored}

...

\subsection{XSS réfléchi (Reflected)}\label{vulnerabilites:web:xss:reflected}

...

\section{Cross-Site Request Forgery (CSRF / XSRF)}\label{vulnerabilites:web:csrf}

Intro vulnérabilités CSRF

\endinput


%%%
 % File:     network.tex
 % Author:   Hackademics Forum <hackademicsforum6@gmail.com>
 % Project:  MindMap des vulnérabilités
 % Released: 03/08/2016
%%%

%!TeX root = main.tex
%!TeX encoding = UTF-8
%!TeX program = pdflatex
%!TeX spellcheck = fr_FR

%%%
 % Vulnérabilités réseau
%%%
\chapter{Vulnérabilités réseau}\label{vulnerabilites:reseau}

Intro vulnérabilités réseau

%%%
 % Inclusion des sections
%%%
%%%
 % File:     spoofing.tex
 % Author:   Hackademics Forum <hackademicsforum6@gmail.com>
 % Project:  MindMap des vulnérabilités
 % Released: 03/08/2016
%%%

%!TeX root = main.tex
%!TeX encoding = UTF-8
%!TeX program = pdflatex
%!TeX spellcheck = fr_FR

%%%
 % Vulnérabilités Spoofing
%%%
\newpage
\section{Usurpation (Spoofing)}\label{vulnerabilites:reseau:spoofing}

Le spoofing est une vulnérabilité qui touche les plus importants protocoles réseau : IP, ARP, MAC, DNS.
Son utilisation permet à un attaquant de se faire passer pour ce qu'il n'est pas, %
afin de cacher ses actions ou de contourner des mécanismes de sécurité.

\subsection{IP Spoofing}\label{vulnerabilites:reseau:spoofing:ip}

Comme son nom l'indique l'IP Spoofing va se situer au niveau du protocole IP. L'attaque consiste à fabriquer et émettre vers une cible des paquets IP avec une fausse adresse d'origine. Cela va avoir deux conséquences :
\\
\begin{itemize}
\item La cible va envoyer un ou des paquets de réponse à la machine dont l'adresse IP a été spoofée.
\item Si la cible utilise l'adresse IP d'origine pour filtrer les accès, elle va accepter des paquets IP qu'elle aurait dû refuser.
\end{itemize}

\begin{tabbing}
\end{tabbing}
L'IP Spoofing permet les types d'attaques suivants :\\

\begin{itemize}
\item Déni de service
\item Prise de controle
\end{itemize}

\subsubsection{Déni de service}\label{vulnerabilites:reseau:spoofing:ip:dos}

Actuellement le type d'attaque par IP Spoofing le plus utilisé est le déni de service distribué (en anglais DDOS / Distributed Denial Of Service). C'est le cas d'utilisation d'IP Spoofing le plus simple :\\
\begin{itemize}
\item L'attaquant va envoyer des quantités énormes de paquets IP vers des destinations diverses, en indiquant comme adresse IP d'origine l'adresse de la cible qu'il veut saturer.
\item Les destinataires des paquets vont renvoyer des réponses vers l'adresse qu'ils croient à l'origine.
\item Si le nombre de paquets réponses est suffisamment important, la cible, ou l'infra-structure qui l'héberge voit sa bande passante ou sa capacité de traitement saturée, ce qui rend ses services injoignables pour les utilisations légitimes.
\end{itemize}

\begin{figure}[hbtp]
\caption{Déni de service}
\centering
\includegraphics[scale=1]{../images/ip-spoofing-ddos.png}
\end{figure}

\begin{tabbing}
\end{tabbing}
Il existe divers outils pour créer de tels paquets IP. nmap est un des plus connus. Dans cet exemple, l'attaquant pourrait exécuter une commande semblable à :\\
\begin{center}
nmap -S 10.0.0.253 -e wlan0 -Pn 10.0.0.0/24
\end{center}


\subsubsection{Prise de controle}\label{vulnerabilites:reseau:spoofing:ip:control}

Certains types de services utilisent l'adresse IP des clients pour controler les accès. C'est le cas notamment des services rsh ou rlogin, qui servent à prendre à distance le controle d'un serveur. Ce type d'attaque tend à être de moins en moins utilisable, les services peu sécurisés comme rlogin et rsh n'étant quasimment plus utilisés.
\\

Contrairement à l'attaque précédente, qui était relativement simple, celle-ci va être plus complexe à mettre en oeuvre. Elle nécessite en effet de connaître la ou les adresses IP autorisée(s) par la cible, et elle ne peut pas être exécutée en aveugle. Il faudra donc la coupler avec d'autres attaques afin de détourner le trafic destiné à l'adresse IP usurpée.
\\

Les différentes étapes de cette attaque seront donc :

\begin{itemize}
\item Trouver l'adresse IP autorisée par la cible
\item Mettre hors service la machine autorisée (par un DOS par exemple) pour l'empêcher de répondre aux paquets éventuellement envoyés par la cible.
\item Prédire les numéros de séquence TCP de la cible.
\item Lancer l'attaque : envoyer à la cible un paquet TCP avec le flag SYN, sur le service ciblé.
\item La cible va répondre avec un paquet SYN-ACK.
\item Répondre à la cible avec un paquet ACK et le bon numéro de séquence TCP.
\item La connexion TCP est alors établie.
\item Envoyer un paquet PSH (remontant les données directement à l'application) pour envoyer une commande au service (exemple : echo ++ >> /.rhosts).
\end{itemize}
\begin{tabbing}
\end{tabbing}

L'attaque peut donc se résumer ainsi (ici, la machine A est celle de l'attaquant, C la machine autorisée, A(C) indiquant que A spoofe l'adresse de C) :

\begin{figure}[hbtp]
\caption{Prise de controle}
\centering
\includegraphics[scale=1]{../images/ip-spoofing-control.png}
\end{figure}

\subsection{ARP Spoofing}\label{vulnerabilites:reseau:spoofing:arp}

...

\subsection{MAC Spoofing}\label{vulnerabilites:reseau:spoofing:mac}

...

\subsection{DNS Spoofing}\label{vulnerabilites:reseau:spoofing:dns}

...

\endinput

%%%
 % File:     replay.tex
 % Author:   Hackademics Forum <hackademicsforum6@gmail.com>
 % Project:  MindMap des vulnérabilités
 % Released: 03/08/2016
%%%

%!TeX root = main.tex
%!TeX encoding = UTF-8
%!TeX program = pdflatex
%!TeX spellcheck = fr_FR

%%%
 % Vulnérabilités Replay
%%%
\newpage
\section{Rejeu (Replay / Playback)}\label{vulnerabilites:reseau:replay}

Une attaque par rejeu ou "Replay Attack" est une attaque de type "Man In The Middle". Cette attaque permet à un pirate d'espionner les données échangées entre deux ordinateurs afin de capturer les paquets et d'enregistrer des données utiles et confidentielles d'authentification qui pourront ensuite être réutilisées malgré le chiffrement pour être soumises à un ordinateur (client/serveur) afin être rejoués tel quels, comme par exemple des transactions bancaires. Mais cette technique ne se limite pas aux banques et peut être utilisé pour voler des informations de connexion de boite e-mail, voler votre numéro de carte bancaire ou tout type de données personnelle qui sera utilisé par le/les pirates pour leurs activité illégales. Le rejeu permettra de faire croire en une connexion légitime et pourra permettre au pirate de s'authentifier et avoir accès au système en usurpant l'identité de la victime. 

\begin{flushleft}
Pour capturer les paquets, le hacker pourra utiliser un script fait maison ou alors plus simplement des logiciels comme wireshark, tcpdump ou zed attack proxy. Ces paquets seront généralement enregistrés au format pcap. Il utilisera ensuite soit à nouveau un script fait maison ou alors un logiciel de type tcpreplay pour renvoyer (rejouer) ces paquets afin de se faire passer comme quelqu'un de légitime.
\end{flushleft}

\bigskip

\begin{itemize}
\item Des informations utiles sont envoyées a travers le réseau
\item Un hacker pourra utiliser ces informations afin de les réutiliser a mauvais escient
\item Il aura besoin d'un accès aux données qui se fera
\begin{itemize}
\item Physiquement
\item MITM 
\item Malware
\end{itemize}
\item Les informations récupérées seront utiles au pirate
\item Qui pourra les réutiliser en les rejouant pou paraître comme légitime
\end{itemize}

\bigskip

\begin{flushleft}
\textbf{Exemple}
\end{flushleft}

\smallskip

\begin{flushleft}
Supposons une communication entre deux ordinateurs A et B. A va envoyer sa clef à B pour prouver son identité. Mais un pirate C est en train d'espionner cette conversation et décide de garder les informations ainsi obtenues dans la conversation entre A et B. Ces informations lui seront utiles pour se faire passer comme légitime en rejouant les données de connexion à B
\end{flushleft}

\smallskip

\begin{center}
\includegraphics[scale=0.3]{Network/assets/rejeu.png}
\end{center}

\bigskip


\begin{flushleft}
\textbf{Contre-mesures}
\end{flushleft}

\smallskip

\begin{itemize}
\item Le chiffrement n'étant pas suffisant pour contrer les attaques par rejeu il est
conseillé d'utiliser.
\item L'utilisation d'un nombre arbitraire (NONCE) généré aléatoirement a usage
unique qui sera utilisé pour signer les échanges de données. En utilisation avec
un protocole de challenge-réponse.
\item L'utilisation d'horodatage chiffré est un autre moyen d’empêcher les
attaques par rejeu. La synchronisation devrait être réalisée en utilisant un
protocole sécurisé.
\item L'utilisation de mots de passe a utilisation unique qui expirent après un
certain temps.
\item Un numéro de séquence peut être utilisé. L'authenticité de ce numéro pourra
être vérifiée par un code d'authentification de message.
\item l'utilisation de tokens de session unique en utilisant un processus de
génération aléatoire.
\end{itemize}

\bigskip

\begin{flushleft}
\textbf{Exemple}
\end{flushleft}

\smallskip

\begin{flushleft}
Pour prévenir d'une attaque par rejeu on pourrait calculer le delta (différence) entre l'horodatage du client et celui du serveur. Cette différence ne sera pas autorisée a être plus grande que (par exemple) une minute (mais le delta sera autorisé a changer périodiquement pour autoriser les changements horaires). En complement l'utilisation d'un NONCE de grande taille (on pourrait aussi utiliser un NONCE alphabétique ou alphanumérique). L'utilisation de l'horodatage et du NONCE devraient être protégés par l'implémentation d'un  HMAC (Algorithme Hash-based Message  Authentication Code).
\end{flushleft}

\bigskip

\begin{flushleft}
\textbf{Lien-utiles}
\end{flushleft}

\begin{itemize}
\item \url{https://www.owasp.org/index.php/Testing_for_WS_Replay_(OWASP-WS-007)}
\item \url{https://www.sitepoint.com/how-to-prevent-replay-attacks-on-your-website/}
\item \url{item h222767.temppublish.com/15_NS/NS_lecture9.ppt}
\item \url{https://tools.ietf.org/html/rfc2085}
\item \url{https://tools.ietf.org/html/rfc2289}
\item \url{https://tools.ietf.org/html/rfc7384}
\end{itemize}

\endinput

%%%
 % File:     dos.tex
 % Author:   Hackademics Forum <hackademicsforum6@gmail.com>
 % Project:  MindMap des vulnérabilités
 % Released: 03/08/2016
%%%

%!TeX root = main.tex
%!TeX encoding = UTF-8
%!TeX program = pdflatex
%!TeX spellcheck = fr_FR

%%%
 % Vulnérabilités DoS
%%%
\section{Déni de service (distribué) (DoS / DDoS) }\label{vulnerabilites:reseau:dos}

Intro vulnérabilités DoS / DDoS

\subsection{Advanced persistent DoS (APDoS)}\label{vulnerabilites:reseau:dos:apdos}

...

\subsection{DoS réflechi (Reflected / Spoofed)}\label{vulnerabilites:reseau:dos:reflected}

...

\endinput


\endinput



\newpage

\end{document}