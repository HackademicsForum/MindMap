%%%
 % File:     introduction.tex
 % Author:   Hackademics Forum <hackademicsforum6@gmail.com>
 % Project:  MindMap des vulnérabilités
 % Released: 03/08/2016
%%%

%!TeX root = main.tex
%!TeX encoding = UTF-8
%!TeX program = pdflatex
%!TeX spellcheck = fr_FR

%%%
 % Introduction
%%%
\chapter*{Introduction}\label{introduction}
\addcontentsline{toc}{chapter}{Introduction}

%% Aspects généraux %%

Ce projet a été réalisé par les membres du forum Hackademics.fr.

Son but est de présenter un MindMap des vulnérabilités en sécurité informatique.

Pour ce faire, le projet est composé de deux parties : 
\begin{itemize}
\item Un mind map (visuel) représentant les différents types de vulnérabilités ainsi que les vulnérabilités liées à ces derniers
\item Ce document (écrit) qui a pour but de présenter les différentes vulnérabilités abordées avec quelques informations sur ces dernières.
\end{itemize}

%% Problématique %%

\section{But des différentes parties}

\subsection{La partie visuelle}
Ce mind map doit pouvoir aider tout expert en sécurité informatique à re-localiser une faille particulière dans la pléthore de faille existantes. 
Une fois cette faille et son type localisé, l doit être possible de se référer à ce document.

 \subsection{La partie écrite}
 L'idée générale est que ce document devrait pouvoir être utilisé comme aide-mémoire en cas de besoin.  Son but n'est pas d'être exhaustif mais de pouvoir s'en sortir et se souvenir des informations les plus importantes concernant une faille.

\endinput
