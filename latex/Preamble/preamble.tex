%%%
 % File:     preamble.tex
 % Author:   Hackademics Forum <hackademicsforum6@gmail.com>
 % Project:  MindMap des vulnérabilités
 % Released: 03/08/2016
%%%

%!TeX root = main.tex
%!TeX encoding = UTF-8
%!TeX program = pdflatex
%!TeX spellcheck = fr_FR

%% Paramètres régionnaux et linguistiques %%
\usepackage[T1]{fontenc}      % Encodage supportant les caractères accentués
\usepackage[utf8]{inputenc}   % Encodage des caractères en UTF-8
\usepackage{eurosym}          % Possibilite d'utiliser le symbole euro
\usepackage{datetime}         % Formatage de la date

%% Police d'écriture %%
%\usepackage{uarial}
%\renewcommand{\familydefault}{\sfdefault}

%% Mise en page %%
\newcommand{\HRule}{\rule{\linewidth}{1mm}}  % Interlignes
\linespread{1}       % Interlignes
\usepackage{lscape}    % Permet la bascule en mode paysage/portrait, portrait/paysage
\usepackage{setspace}  % Permet de définir la taille d'interlignage
                       % à l'aide des environnements onehalfspace et doublespace
\usepackage{xspace}    % Espaces intelligents dans les macros
\usepackage[top=2cm,
						bottom=2cm,
						left=2cm,
						right=2cm,
						bindingoffset=7mm,
						includehead,
						includefoot]{geometry}  % Modification du gabarit du document

%% Profondeur dans la table des matières %%
\setcounter{secnumdepth}{4}
\setcounter{tocdepth}{4}

%% Abstract %%
\usepackage{abstract}

%% Gestion des références internes et externes sur un document PDF %%
\usepackage{hyperref}

%% Méta-données %%
\usepackage{ifpdf}

%% Gestion des URI %%
\usepackage{url}

%% Gestion des TODO %%
\usepackage[colorinlistoftodos]{todonotes}

%% Annexes %%
\usepackage[title,toc,page,header]{appendix}
\renewcommand{\appendixtocname}{Annexes}
\renewcommand{\appendixpagename}{Annexes}

%% Chapitres %%
\usepackage{titlesec, blindtext, color}
\titleformat{\chapter}[hang]{\relax}{}{0pt}{\huge}[\titlerule]

%% Éléments graphiques %%
\usepackage{graphicx}
\usepackage{subfig}  %% Gallerie d'images

%% Objets flottants %%
\usepackage{float}
%\usepackage{placeins}
%\usepackage{floatrow}

%% Tableaux %%
\usepackage{array}
\usepackage{tabularx}

%% Bibliographie %%
%\usepackage[numbers,sort&compress]{natbib}

%% Formules mathématiques %%
\usepackage{amsmath}
\usepackage{amsthm}
\usepackage{amsfonts}

\endinput
